The transition to hydrogen-powered transportation faces a critical infrastructure challenge: while hydrogen \gls{fcev} technology has matured to commercial viability, the refueling network remains severely limited compared to conventional fueling stations. This document presents the technical specification for \textbf{Hydrodomus}, a home-based hydrogen generation and storage system designed to address this infrastructure gap by enabling \gls{fcev} owners to produce and store hydrogen fuel at their residence.

The proposed system utilizes \gls{pem} water electrolysis to generate high-purity hydrogen from water and electricity. The produced hydrogen is compressed to the automotive standard pressure of \SI{700}{bar} and stored in a certified composite vessel, from which the user can refuel their vehicle using a standardized SAE J2601-compliant dispensing system.

This document provides comprehensive coverage of the underlying theoretical principles, including electrochemistry, thermodynamics of gas compression, and hydrogen storage physics. The system architecture is presented in detail, with component specifications derived from both commercial availability and technical requirements. Safety considerations and the regulatory certification pathway are thoroughly analyzed to ensure compliance with international standards including ISO 19880, EC 79/2009, and SAE J2601.

The technical analysis demonstrates that home hydrogen generation is feasible with current technology, with system efficiency in the range of 60--70\% from electricity to stored hydrogen. The primary challenges identified include the high capital cost of \SI{700}{bar} compression equipment and storage vessels, as well as the certification requirements for residential hydrogen systems. The document concludes with recommendations for prototype development and a roadmap toward commercial deployment.

\textbf{Keywords:} Hydrogen production, PEM electrolysis, home refueling, FCEV, 700 bar storage, SAE J2601

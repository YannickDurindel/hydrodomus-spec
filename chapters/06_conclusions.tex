This chapter summarizes the key findings of the technical specification and outlines the path forward for prototype development and commercial deployment.

\section{Summary of Findings}

\subsection{Technical Feasibility}

The analysis presented in this document demonstrates that home-based hydrogen generation for \gls{fcev} refueling is technically feasible using currently available technology. The key technical conclusions are:

\textbf{Hydrogen production:}
\begin{itemize}
    \item \Gls{pem} electrolysis is the preferred technology for residential applications
    \item A 2 kW electrolyzer can produce approximately \SI{0.5}{kg} H$_2$ per day
    \item System efficiency of 60--70\% (electricity to stored hydrogen) is achievable
    \item Water consumption is modest ($\sim$\SI{10}{L/day})
\end{itemize}

\textbf{Compression and storage:}
\begin{itemize}
    \item Multi-stage piston compression is required to reach \SI{700}{bar}
    \item Type IV composite vessels provide safe, certified storage
    \item A \SI{10}{L} vessel stores $\sim$\SI{0.4}{kg} H$_2$, sufficient for $\sim$50 km range
\end{itemize}

\textbf{Dispensing:}
\begin{itemize}
    \item SAE J2601 slow-fill protocol is appropriate for home application
    \item Standard ISO 17268 nozzles ensure vehicle compatibility
    \item Fill times of 10--30 minutes are acceptable for overnight use
\end{itemize}

\subsection{Safety Assessment}

The safety analysis indicates that a residential hydrogen system can be operated safely with appropriate design measures:

\begin{itemize}
    \item Multiple independent protection layers mitigate risks
    \item Hydrogen detection and automatic shutdown prevent hazardous accumulation
    \item Certified pressure vessels and relief systems protect against overpressure
    \item Ventilation requirements are achievable in typical garage installations
\end{itemize}

\subsection{Regulatory Pathway}

A clear certification pathway exists through:
\begin{itemize}
    \item ISO standards for hydrogen generation (ISO 22734) and fueling (ISO 19880)
    \item European regulations for pressure equipment and ATEX compliance
    \item SAE standards for vehicle fueling interface
\end{itemize}

The regulatory framework is developing but is sufficiently mature to support product development.

\section{Economic Considerations}

\subsection{Capital Cost Estimate}

Based on component costs identified in Chapter 4:

\begin{table}[htbp]
    \centering
    \caption{Estimated system capital cost}
    \label{tab:capex}
    \begin{tabular}{@{}lr@{}}
        \toprule
        \textbf{Component} & \textbf{Cost (EUR)} \\
        \midrule
        PEM electrolyzer (2 kW) & 8,000--15,000 \\
        Water treatment system & 500--1,000 \\
        Compressor (700 bar) & 5,000--10,000 \\
        Storage vessel (10 L, Type IV) & 2,000--4,000 \\
        Dispensing system & 2,000--4,000 \\
        Control system & 1,500--3,000 \\
        Safety systems & 1,000--2,000 \\
        Installation and commissioning & 2,000--5,000 \\
        \midrule
        \textbf{Total} & \textbf{22,000--44,000} \\
        \bottomrule
    \end{tabular}
\end{table}

\subsection{Operating Costs}

\begin{table}[htbp]
    \centering
    \caption{Estimated operating costs}
    \label{tab:opex}
    \begin{tabular}{@{}lcc@{}}
        \toprule
        \textbf{Item} & \textbf{Consumption} & \textbf{Annual Cost (EUR)} \\
        \midrule
        Electricity (\EUR{0.20}/kWh) & 55 kWh/kg $\times$ 100 kg/yr & 1,100 \\
        Water & 10 L/kg $\times$ 100 kg/yr & 20 \\
        Maintenance & -- & 300--500 \\
        \midrule
        \textbf{Total} & & \textbf{1,400--1,600} \\
        \bottomrule
    \end{tabular}
\end{table}

\subsection{Comparison with Station Refueling}

At current hydrogen station prices of \EUR{10--15}/kg:

\begin{itemize}
    \item Annual fuel cost at station: \EUR{1,000--1,500} for 100 kg/year
    \item Annual fuel cost with Hydrodomus: \EUR{1,100} (electricity only)
    \item Simple payback on operating cost savings: Not favorable
\end{itemize}

The economic case improves significantly with:
\begin{itemize}
    \item Lower electricity prices (e.g., self-generated solar)
    \item Higher hydrogen station prices (price volatility)
    \item Value of convenience and independence
    \item Future reductions in equipment cost
\end{itemize}

\section{Prototype Development Plan}

\subsection{Phase 1: Laboratory Prototype}

\textbf{Objectives:}
\begin{itemize}
    \item Validate component integration
    \item Characterize system performance
    \item Identify design improvements
\end{itemize}

\textbf{Scope:}
\begin{itemize}
    \item Commercial electrolyzer integration
    \item Low-pressure testing (<30 bar)
    \item Manual operation
    \item Laboratory environment only
\end{itemize}

\subsection{Phase 2: High-Pressure Prototype}

\textbf{Objectives:}
\begin{itemize}
    \item Demonstrate full pressure capability
    \item Validate safety systems
    \item Conduct vehicle refueling tests
\end{itemize}

\textbf{Scope:}
\begin{itemize}
    \item Integration of 700 bar compressor
    \item Certified storage vessel
    \item Automated control system
    \item SAE J2601 compliant dispensing
\end{itemize}

\subsection{Phase 3: Field Trials}

\textbf{Objectives:}
\begin{itemize}
    \item Validate real-world performance
    \item Gather user feedback
    \item Support certification process
\end{itemize}

\textbf{Scope:}
\begin{itemize}
    \item Installation at 3--5 pilot sites
    \item 12-month operational period
    \item Performance monitoring and data collection
    \item Regulatory engagement
\end{itemize}

\section{Recommendations}

\subsection{Technical Recommendations}

\begin{enumerate}
    \item \textbf{Prioritize elevated-pressure electrolysis}: Select an electrolyzer with >30 bar output to reduce compression requirements and cost

    \item \textbf{Consider electrochemical compression}: Emerging technology that could replace mechanical compressors with lower cost and maintenance

    \item \textbf{Design for solar integration}: Include capability for direct DC coupling to photovoltaic systems

    \item \textbf{Implement remote monitoring}: Enable proactive maintenance and safety monitoring through cloud connectivity
\end{enumerate}

\subsection{Business Recommendations}

\begin{enumerate}
    \item \textbf{Partner with vehicle manufacturers}: Collaboration with Toyota, Hyundai, or BMW could provide market access and technical validation

    \item \textbf{Target early adopter markets}: Focus on regions with limited hydrogen infrastructure and high environmental awareness

    \item \textbf{Explore lease/subscription models}: Reduce upfront cost barrier through alternative ownership models

    \item \textbf{Seek regulatory engagement}: Proactive engagement with authorities to clarify permitting requirements
\end{enumerate}

\section{Conclusion}

The Hydrodomus home hydrogen generation system represents a viable approach to addressing the infrastructure barrier facing hydrogen mobility. While capital costs remain high, the technology is mature and the regulatory pathway is defined. As \gls{fcev} adoption grows and component costs decrease with volume production, home hydrogen generation could become an attractive option for environmentally conscious consumers seeking energy independence.

The immediate next step is to proceed with laboratory prototype development to validate the integrated system concept and refine the technical specifications. Success at this stage will provide the foundation for subsequent high-pressure development and eventual commercialization.
